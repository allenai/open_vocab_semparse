\title{Cover Letter: \\ Learning a Compositional Semantics for Freebase \\ with an Open
  Predicate Vocabulary} 

\documentclass[10pt]{article}
\usepackage[margin=1.0in]{geometry}
\usepackage{mdwlist}


\begin{document}
\maketitle

We appreciate the helpful comments from the reviewers. Our responses
to the three specific changes are listed below:

\begin{itemize*}

\item {\bf Comparison to Reddy et al. (2014):} We have added a
  discussion comparing our approach to this one in Section 8 (Related
  Work). This work is similar to ours in its training data
  generation. It differs from our work in that it uses a closed
  predicate vocabulary -- as do all semantic parsers to
  Freebase. (This difference was acknowledged by the reviewers in
  their original reviews.)

\item {\bf Coverage and Expressivity of Approach:} We have added a
  Discussion (Section 3.4) to the semantic parsing portion of the
  paper describing our system's coverage of linguistic phenomena. Note
  that we have chosen to drop negation from the description of the
  semantic parsing system, as it is (1) confusing to the readers, and
  (2) irrelevant for both the training and evaluation.

\item {\bf Comparing Probabilistic Databases to Possible Worlds
  Semantics:} We have changed our description of the system to use
  possible worlds semantics throughout. Section 2 (System Overview)
  sets up a probabilistic model for the whole system, and Section 4
  (Probabilistic Database) explains how a probabilistic database
  represents a probability distribution over possible worlds. Section
  5 (Inference) has also been improved to explain how the recursive
  query evaluation algorithm is consistent with the possible worlds
  semantics. Finally, Section 8 briefly compares probabilistic
  databases to Markov Logic Networks -- we believe the relationship
  does not require much explanation, given the improved presentation.

\end{itemize*}

We have also made changes based on the additional suggestions of the
reviewers. These are fairly minor changes, but we point them out to
assist in the re-review:

{\bf Reviewer 1}

The changes mentioned above address the comments about the knowledge
represented in the probabilistic database, and the relationship of
probabilistic databases to MLNs. The discussion of
coverage/expressivity also cites Lewis and Steedman (2013). Finally,
Figure 3 now uses human-readable IDs.

% We have not changed the text describing the clustering baseline,
% because the clusters look reasonable on manual inspection. Also, note
% that our training set is approximately 10x larger than the comparable
% work of Riedel et al. (2013).

{\bf Reviewer 2}

The new presentation of the probabilistic database and explicit
discussion of the coverage of our semantic parsing approach should
address many of your concerns. This presentation lays out a general
probabilistic model for thinking about compositional semantics with an
open predicate vocabulary, even though our system only implements a
small piece of it. We also explain the extraction of simplified
logical forms, which we believe also clarifies the ranking objective
explanation. Footnote 4 addresses your suggestion of training with
randomly-generated negative examples (which we tried, and found
performed poorly). Finally, the discussion section now includes a
high-level error analysis.


{\bf Reviewer 3}

Most of your comments should be addressed by the 3 changes
above. Additionally, Table 1 has been clarified as suggested, and the
evaluation discusses how the hyperparameters were selected. Finally,
the discussion section has been considerably improved.

(Also, thanks for the LaTeX tip.)

\end{document}

