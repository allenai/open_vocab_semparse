\documentclass[11pt]{article}

\usepackage{booktabs}
\usepackage[pdf]{pstricks}
\usepackage{pst-pdf}
\usepackage{pst-plot}
\usepackage{times}
\usepackage{multirow}
\usepackage{amsmath}
\usepackage{latexsym}
\usepackage{soul}
\usepackage{subfig}
\usepackage{stmaryrd}

\title{Learning a Compositional Semantics for Freebase \\ with an Open
  Predicate Vocabulary}

% \author{Author 1\\
% 	    XYZ Company\\
% 	    111 Anywhere Street\\
% 	    Mytown, NY 10000, USA\\
% 	    {\tt author1@xyz.org}
% 	  \And
% 	Author 2\\
%   	ABC University\\
%   	900 Main Street\\
%   	Ourcity, PQ, Canada A1A 1T2\\
%   {\tt author2@abc.ca}}

\date{}

% Contributions:
% - compositional model of semantics based on universal schema (cite old work)
% - new training objective
% - experimental evaluation
% 
% Introduction:
% 
% - Fixed schema is inherently limiting
% - open / universal is good
%   - need techniques for composition / compositional semantics
% - 

\newcommand{\pred}[1]{\textsc{#1}}
% TODO: make this look cleaner!
\newcommand{\ccgsyn}[1]{\mbox{$\it{#1}$}}
\newcommand{\true}{\pred{True}}

\newcommand{\clb}{\textsc{CorpusLookup}}
\newcommand{\cb}{\textsc{Clustering}}
\newcommand{\uspr}{\textsc{Factorization} (\ensuremath{O_P})}
\newcommand{\usqr}{\textsc{Factorization} (\ensuremath{O_Q})}
\newcommand{\epr}{\textsc{Ensemble} (\ensuremath{O_P})}
\newcommand{\eqr}{\textsc{Ensemble} (\ensuremath{O_Q})}
\newcommand{\pden}[1]{\ensuremath{\llbracket #1 \rrbracket_{P}}}

\newcommand{\rightedge}[1]{\ensuremath{\xRightarrow[]{#1}}}
\newcommand{\leftedge}[1]{\ensuremath{\xLeftarrow[]{#1}}}
\newcommand{\biedge}[1]{\ensuremath{\xLeftrightarrow[]{#1}}}
\newcommand{\entitysentence}{$(e_1, e_2, s)$}
\newcommand{\entsents}{\ensuremath{S_{(e_1, e_2)}}}

\newcommand{\bS}{\textbf{S}}
\newcommand{\bs}{\textbf{s}}
\newcommand{\bY}{\textbf{Y}}
\newcommand{\by}{\textbf{y}}
\newcommand{\bZ}{\textbf{Z}}
\newcommand{\bz}{\textbf{z}}
\newcommand{\bell}{\boldsymbol \ell}
\newcommand{\bEll}{\textbf{L}}
\newcommand{\bT}{\textbf{T}}
\newcommand{\boldt}{\textbf{t}}
\newcommand{\vecell}{\boldsymbol \ell}
\newcommand{\btheta}{\ensuremath{\textbf{\theta}}}

\newcommand{\km}{\textsc{Asp}}
\newcommand{\kmsyn}{\textsc{Asp-syn}}
\newcommand{\kmnobackoff}{\textsc{-backoff}}
\newcommand{\kmnoccgbank}{\textsc{-ccgbank}}
\newcommand{\kmold}{\textsc{K\&M-2012}}
\newcommand{\pipeline}{\textsc{Pipeline}}

\newcommand{\mention}{\pred{m}}
\newcommand{\mentionspan}[2]{#1}

\newcommand{\fsyn}{\ensuremath{O_{\mbox{syn}}}}
\newcommand{\fsem}{\ensuremath{O_{\mbox{sem}}}}
\newcommand{\bt}{\textbf{t}}
\newcommand{\bd}{\boldsymbol d}


\begin{document}

\begin{eqnarray*}
\ell =  \\
s =   \\
t = \\
\end{eqnarray*}

\begin{eqnarray*}
\fsem{}(\theta)  = \sum_{(e_1, e_2)} | \max_{\hat{\bell}, \hat{\bt}} \Gamma(\hat{\bell},\hat{\bt{}},\bS{}_{(e_1, e_2)}; \theta)  - \max_{\bell^*, \bt^*} \big( \Psi(\by_{(e_1, e_2)}, \bell^*) + \Gamma(\bell^*,\bt{}^*,\bS{}_{(e_1, e_2)} ;\theta)\big) |_{+}\\
\end{eqnarray*}

\begin{eqnarray*}
\fsyn{}(\theta) & = & \sum_{i=1}^n |\max_{\hat{\ell}, \hat{t}} \Gamma(\hat{\ell}, \hat{t} , s_i; \theta) - \max_{\ell^*} \Gamma(\ell^*, t_i , s_i; \theta)|_{+} \\
\end{eqnarray*}

$O(\theta) = \fsyn{}(\theta) + \fsem{}(\theta)$


\begin{eqnarray*}
P(\gamma | s) = \sum_K \sum_\ell P(\gamma | \ell, K) P(K) P(\ell | s) 
\end{eqnarray*}


\begin{eqnarray*}
P(K) & = & \prod_{e \in E} \prod_{c \in C} P( c(e) = K_{c,e}) \times \\
 & & \prod_{e_1\in E} \prod_{e_2 \in E} \prod_{r \in R} P(r(e_1, e_2) = K_{r, e_1, e_2})
\end{eqnarray*}


\begin{eqnarray*}
P(e \in \gamma | s) & = & \sum_\gamma \textbf{1}(e \in \gamma) P(\gamma | s) \\
& = & \sum_K \textbf{1}(e \in \llbracket \ell \rrbracket_K) P(K)
\end{eqnarray*}

\begin{eqnarray*}
\theta_{prs}^{j+1} \leftarrow \theta_{prs}^j + \alpha_j ( \phi_{prs}(\ell^*, t^*, s^i) - \phi_{prs} (\hat{\ell}, \hat{t}, s^i))
\end{eqnarray*}


\begin{eqnarray*}
\hat{\ell}, \hat{t} & \leftarrow & \arg \max_{\ell, t} \Gamma(\ell, t , s_i; \theta^t) \\
\ell^* & \leftarrow & \arg \max_{\ell} \Gamma(\ell, t_i , s_i; \theta^t) \\
\theta^{t+1} & \leftarrow & \theta^t + \phi(\ell^*, t_i, s_i) - \phi(\hat{\ell}, \hat{t}, s_i)
\end{eqnarray*}

\begin{eqnarray*}
\hat{\bell}, \hat{\bt} & \leftarrow & \arg \max_{\bell, \bt} \Gamma(\bell, \bt , \bS{}_{(e_1, e_2)}; \theta^t) \\
\bell^*, \bt^* & \leftarrow & \arg \max_{\bell, \bt} \Gamma(\bell, \bt , \bS{}_{(e_1, e_2)}; \theta^t) + \Psi(\by_{(e_1, e_2)}, \bell) \\
\theta^{t+1} & \leftarrow & \theta^t + \phi(\bell^*, \bt^*, \bS{}_{(e_1, e_2)}) - \phi(\hat{\bell}, \hat{\bt}, \bS{}_{(e_1, e_2)}) \\
\end{eqnarray*}

\begin{figure}
\center
\includegraphics[width=5in]{figures/open_vocabulary/ap_plot.pdf}
\vspace{-1in}
\caption{Averaged 11-point precision/recall curves for the 116
  answerable test questions.}
\vspace{-.1in}
\label{fig:pr-curve}
\end{figure}



\end{document}
